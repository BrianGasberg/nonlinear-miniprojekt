The trajectory reaches the boundary layer near the sliding surface where $|s| \leq \epsilon$ in finite time. Inside the boundary layer the system dynamics reduces to $\dot{x}_1 = -a_1 x_1 + s$. In order to insure performance of the sliding mode control the Lyapunov stability in the sliding phase is analysed by the Lyapunov candidate $V = (1/2) x_1^2$. The derivative is then given as:
\begin{equation}
        \begin{split}
                \dot{V}_1(x) &= x_1 \dot{x}_1 = -a_1 x_1^2 + x_1 s \leq -a_1 x_1^2 + |x_1| \epsilon \\
                &\leq -(1-\theta)a_1 x_1^2 - \theta a_1|x_1|^2 + |x_1|\epsilon \leq 0 \quad \forall \quad |x_1| \geq \frac{\epsilon}{a_1 \theta}
        \end{split}
\end{equation}
where $0 < \theta < 1$. This means that when the trajectory reaches the set:
\begin{equation}
        \Omega_\up{\epsilon} = \left\{|x_1| \leq \frac{\epsilon}{a_1 \theta}, \quad |s| \leq \epsilon \right\}
\end{equation}
it stays inside $\Omega_\up{\epsilon}$. 
\begin{equation}
        |x_2| = |a_1 x_1 + x_2 - a_1 x_1| \leq |a_1 x_1 + x_2| - |a_1 x_1| \leq \left(1 - \frac{1}{\theta}\right)\epsilon
\end{equation}

Where we have chosen $\theta = 0.94$ where the norm of the angular velocity is then given as $|x_2| \leq 0.0638 \g \epsilon \leq 0.01 \, \text{rad/s}$. We should then choose $\epsilon$ small enough to insure the performance requirement of $0.01$ rad/s.
\begin{equation}
         \epsilon = 0.157 \nonumber
 \end{equation} 
 The results is shown in Figure \ref{fig:smc} with comments in Appendix \ref{cha:sliding_mode}.

% section bound (end)
