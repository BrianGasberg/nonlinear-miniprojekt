\section{Bound} % (fold)
\label{sec:bound}

\begin{equation}
        \begin{split}
                \dot{V}_1(x) &= x_1 \dot{x}_1 = -a_1 x_1^2 + x_1 s \leq -a_1 x_1^2 + |x_1| \epsilon \\
                &\leq -(1-\theta)a_1 x_1^2 - \theta a_1|x_1|^2 + |x_1|\epsilon \leq 0 \quad \forall \quad |x_1| \geq \frac{\epsilon}{a_1 \theta}
        \end{split}
\end{equation}
The bound is then defined as:
\begin{equation}
  \Omega = \{ |x_1| \leq \frac{\epsilon}{-a_1 \g \theta} \quad , \quad |x_1 + x_2| \leq \epsilon \}
\end{equation}

\begin{equation}
  |x_2| = |x_1 + x_2 - x_1| \leq |x_1 + x_2| - |x_1| \leq (1 - \frac{1}{a_1 \g \theta}) \g \epsilon
\end{equation}
$\theta$ is used for testing the stability of the Lyaponov function, and should be sent as $0 \leq \theta \leq 1$. $a_1$ is found through simulation, and determines the slope of the sliding surface. The slope must be positive.
\begin{equation}
  \theta = 0.9 \quad a_1 = 8
\end{equation}

\begin{equation}
  \begin{split}
    (1-\frac{1}{a_1 \g \theta}) \g \epsilon \leq 0.01 \\
    \epsilon < \frac{0.01 \g a_1 \g \theta}{a_1 \g \theta - 1} \\
    \epsilon < 0.0116
  \end{split}
\end{equation}

% It can be shown that the reaching and sliding surfaces within a given bound will guarantee convergence. The bound can be found by looking at the system and its Lyapunov function.



% Define $\rho$ such that:
% \begin{equation}
%         \left \vert \frac{a_1 x_2 + h(x)}{g(x)} \right \vert \leq \rho(x) 
% \end{equation}

% And choose a lyapunov function candidate:
% \begin{equation}
%         V=(1/2)s^2
% \end{equation}
% for:
% \begin{equation}
%         s=a_1 x_1+ x_2 = 0
% \end{equation}
% The derived of the Lyapunov function is found by:
% \begin{equation}
%         \begin{split}
%                 \dot{s} &= a_1 \g \dot{x}_1 + \dot{x}_2 \\
%                 \dot{s} &= a_1 \g x_2 + h(x) + g(x) \g u\\
%                 \dot{V} &= s \g \dot{s} = s(a_1 \g x_2+ h(x)) + g(x) \g s \g u \leq g(x) \g \vert s \vert \g \rho(x) + g(x) \g s\g u
%         \end{split}
% \end{equation}
        

% The upper bound of $\epsilon$ is found via:
% \begin{equation}
%         \begin{split}
%                 \frac{(m \g l \g g_0)}{a_1} &\leq \frac{k}{\epsilon} \\
%                 \epsilon &\leq \frac{k \g a_1}{m \g l \g g_0}    
%         \end{split}
% \end{equation}

% section bound (end)
