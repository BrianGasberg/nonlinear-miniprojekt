Sliding mode control is a non-linear control design method. The sliding mode controller forces the system trajectories to
reach a sliding surface in finite time. This sliding surface is called a sliding manifold. When the system is sliding
along the sliding manifold the system dynamics are eliminated, and the system cannot leave the manifold. The sliding
manifold then forces the system trajectories to reach an equilibrium, and stay there. The reference of the control
system is then the equilibrium. 

In order to control the system and be able to reach the sliding manifold the control strategy is to design the behaviour
in to different phases, where the control gain changes based on where the system state are located at any given time in
the state space for the system. When the system state is not at the sliding surface, then it is said to be in the
reaching phase and the control gain is then chosen to be based on a signum function, resulting in a discontinuous control. When the system state is within some defined distance from the sliding manifold, then it is said to be in the
sliding phase and, the controller changes into a continuous controller and at the same time
eliminating the system dynamics. This ensures that the system stays on the sliding manifold leading the states through
the state space and reach an equilibrium of choice.

The purpose of the reaching phase is to get to the manifold of the surface $s$ where:
\begin{equation}
  s = a_1 x_1 + x_2 = 0
\end{equation}
On the manifold the control loop becomes $ \dot{x}_1 = - a_1 x_1$ which is stable for $ a_1 > 0$.
By differentiating and using $ \dot{x}_1 = x_2 $ it becomes:
% \begin{equation}
%   \dot{s} = a_1 x_2 - a sin(x_1 + \pi) - b x_2 + c \left[u(t) + \delta(t) \right] + \xi (t)
% \end{equation}
\begin{equation}
        \dot{s} = a_1 x_2 + \dot{x}_2 = a_1 x_2 + h(x) + g(x)u
\end{equation}
where $h(x)$ and $g(x)$ is the system model. For $\dot{s} = 0$ then the inequality must be satisfied.
\begin{equation}
        \left\vert \frac{a_1 x_2 + h(x)}{g(x)} \right\vert \leq \varrho(x)
\end{equation}
where $\varrho(x)$ is some function. If we choose a Lyapunov candidate for $\dot{s}$ to be $V = (1/2)s^2$, the derivative is then given as:
\begin{equation}
  \dot{V} = s \dot{s} =  s[a_1 x_2 + h(x)] + s g(x) u \leq g(x) \vert s\vert \varrho(x) + g(x) s u \leq 0
\end{equation}
In order to insure Lyapunov stability at the equilibrium point the inequality should hold for $\det{V} \leq 0$. We then take $u(t)$ to be
\begin{equation}
  u(t) = -\beta(x) \up{sgn}(s)
  \label{eq:control_ideal}
\end{equation}
where
\begin{equation}
  \beta (x) \geq \varrho (x) + \beta_0
\end{equation}
The result of the control is that when the trajectories will reach the manifold $s=0$ in a finite time. Ones on the manifold the trajectories can not leave it again. This is seen from the inequality $\dot{V} \leq -g_0 \beta_0 \vert s \vert$. In other words, when the trajectories is on the manifold the change is energy is less than or equal to zero. This is of course only possible in the ideal sliding mode control. 

% Which can be bounded by:
% \begin{equation}
%   \begin{split}
%     \beta (x) \geq \left\Vert \frac{a_1}{c} - \frac{a_1}{\hat{c}} -  \frac{b_1}{c} - \frac{\hat{b}}{\hat{c}}  \right\Vert \g \left\Vert x_2 \right\Vert + \left\Vert \frac{a}{c} - \frac{\hat{a}}{\hat{c}}  \right\Vert \g \left\Vert sin(x_1) \right\Vert + \left\Vert \frac{\delta (t)}{c} \right\Vert + \beta_0 \\
%     = 0.6126 |sin(x_1)| + 0.0147 |x_2| + 0.3371 + \beta_0
%   \end{split}
% \end{equation}