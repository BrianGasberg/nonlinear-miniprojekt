<<<<<<< HEAD
Sliding mode control is a nonlinear control design method. The sliding mode controller forces the system trajectories to
reach a sliding surface in finite time. This sliding surface is called a sliding manifold. When the system is sliding
along the sliding manifold the system dynamics are eliminated, and the system cannot leave the manifold. The sliding
manifold then forces the system trajectories to reach an equillibrium, and stay there. The reference of the control
system is then the equillibrium. 

In order to control the system and be able to reach the sliding manifold the control strategy is to design the behavior
in to different phases, where the control gain changes based on where the system state are located at any given time in
the state space for the system. When the system state is not at the sliding surface, them it is said to be in the
reaching phase and the control gain is then chosen to be based on a signum function, resulting in an agressive
behavior. When the system state is within some defined distance from the sliding manifold, the it is said to be in the
sliding phase and, the controller changes into something causing a less agressive behavior and at the same time
eliminating the system dynamics. This ensures that the system stays on the sliding manifold leading the states through
the state space and reach an qequillibrium of choice.

=======
A sliding mode controller for the pendulum described in the previous section has to be designed. 
>>>>>>> 86565617fbcb72ea631d82c2a755f84287b04393
The purpose of the reaching phase is to get to the manifold of the surface $s $ where:
\begin{equation}
s = a_1 x_1 + x_2 = 0
\end{equation}
On the manifold the control loop becomes $ \dot{x}_1 = - a_1 x_1   $ which is stable for $ a_1 > 0$
By differentiating and using $ \dot{x}_1 = x_2 $ it becomes:
\begin{equation}
\dot{s} = a_1 x_2 - a sin(x_1 + \pi) - b x_2 + c T + \delta (t)
\end{equation}
%For now it will be written as: 
%\begin{equation}
%\dot{s} = a_1 \cdot x_2 + h(x) + g(x)\g u;
%\end{equation}
With the lyapunov function $V = \frac{1}{2} s^2 $ for $ s $ the system is stable for:
\begin{equation}
 \dot{V} = s \dot{s} =  s (a_1 x_2 - a sin(x_1 + \pi) - b x_2  + \delta (t) ) + s c T
\end{equation}
Which holds for
\begin{equation}
T = u = -\beta (x) \g sign(x)
\end{equation}
where
\begin{equation}
\beta (x) \geq \left| \frac{a_1 \cdot x_2 - a \g sin(x_1 + \pi) - b \g x_2  + \delta (t)}{c} \right| + \beta_0
\end{equation}
Which can be bounded by:
\begin{equation}
	\begin{split}
\beta (x) \geq \left\Vert \frac{a_1}{c} - \frac{a_1}{\hat{c}} -  \frac{b_1}{c} - \frac{\hat{b}}{\hat{c}}  \right\Vert \g \left\Vert x_2 \right\Vert + \left\Vert \frac{a}{c} - \frac{\hat{a}}{\hat{c}}  \right\Vert \g \left\Vert sin(x_1) \right\Vert + \left\Vert \frac{\delta (t)}{c} \right\Vert + \beta_0 \\
= 0.6126 |sin(x_1)| + 0.0147 |x_2| + 0.3371 + \beta_0
\end{split}
\end{equation}
