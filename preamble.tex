\documentclass[10pt,a4paper,oneside,openany,article]{memoir}
\usepackage[english]{babel}
\usepackage[utf8]{inputenc}
\usepackage[T1]{fontenc}
\usepackage{amsmath,amssymb,mathtools,bm} % Diverse matematik
\usepackage{mathptmx}
\usepackage{graphicx}
\graphicspath{{figures/}}
\usepackage{pdfpages}
\usepackage{epstopdf}
\usepackage{fullpage}
\usepackage{float}
\usepackage{caption}
\usepackage{color,xcolor}
\usepackage{listings} % Præsentering af kode
\usepackage{pgfplots}
\usepackage{tikzpagenodes}
\usepackage{tkz-euclide}
\usepackage[acronym,toc,nonumberlist]{glossaries}
\usepackage[pdftitle={Nonlinear Control Systems - Mini Project},
            pdfauthor={{Brian Gasberg Thomsen, Jens Nielsen, Mikael Juhl Kristensen and Nikolaj Holm},
               Aalborg University,
               Aalborg,
               Denmark},
            pdfduplex=DuplexFlipLongEdge,
            colorlinks=false,
            hidelinks=true,
            pdftex,
            pdfmenubar=true,
            pdftoolbar=true,
            pdfstartview={FitH}
            ]{hyperref}
\usepackage{memhfixc}
\usepackage{tikz}
\usetikzlibrary{arrows,shapes,backgrounds,patterns,decorations.pathreplacing, decorations.pathmorphing,decorations.markings,shadows,shapes.misc,calc, positioning}
\captionsetup{font=footnotesize,labelfont={bf},textfont=normalfont}

\setlength{\parindent}{1.5mm} % Størrelse af indryk ved nyt afsnit
\setlength{\parskip}{1.8mm} % Afstand mellem afsnit ved brug af "double Enter"

\newlength\figureheight 
\newlength\figurewidth 

% Farver til listset
\definecolor{dkgreen}{rgb}{0,0.6,0} % Definerer farven grøn til brug i præsentation af kildekode
\definecolor{gray}{rgb}{0.5,0.5,0.5} % Definerer farven grå til brug i præsentation af kildekode
\definecolor{mauve}{rgb}{0.58,0,0.82} % Definerer farven pink til brug i præsentation af kildekode
\definecolor{darkblue}{rgb}{0.0,0.0,0.6}

% Listset til præsentation af kildekode
\lstset{frame=tb,
  language=C,
  aboveskip=3mm,
  belowskip=3mm,
  showstringspaces=false,
  columns=flexible,
  basicstyle={\footnotesize\ttfamily}, % basicstyle={\small\ttfamily},
  stepnumber=2,
  firstnumber=1,
  escapeinside={@}{@},
  numberfirstline=false,
  numbers=left,
  numbersep=8pt,
  numberstyle=\tiny\color{gray},
  keywordstyle=\color{blue},
  commentstyle=\color{dkgreen},
  stringstyle=\color{mauve},
  breaklines=true,
  breakatwhitespace=true
  tabsize=3,
  captionpos=b,
  keepspaces=true
}
