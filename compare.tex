\section{Results and Conclusion} % (fold)
\label{sec:results_and_conclusion}
The following controller has be found through simulations
\begin{equation}
\begin{split}
        \delta_1 &= \pi \\
        \varepsilon &= 0.157\\
        a_1 &= 15 \\
        u &=  -\left[ 1576 \vert \bar{x}_1-\delta_1 \vert + 615.2\vert x_2 \vert + 422 \right]\up{sat}\left(\frac{s}{\varepsilon}\right)\nonumber \\
\end{split}
\end{equation}

\subsection{Sliding Mode Control vs. Lyapunov Redesign} % (fold)
\label{sub:sliding_mode_control_vs_lyapunov_redesign}
In Appendix \ref{cha:sliding_mode} and Appendix \ref{cha:lyapunov_redesign_results}, plots of simulation results of both
control design via Lyapunov redesign and Sliding mode. Figure \ref{fig:lyap} is depicting the step response of the
simulation results of a Lyapunov redesign control design technique. Figure \ref{fig:smc} is depicting the step response
of the simulation results of a Sliding mode control design technique. Both control designs meets the requirements for the
system.

It can be seen on the figures that the sliding mode controller is faster at reaching the target angle. This comes from the sliding mode being more aggressive than the Lyaponov redesign controller. If the disturbance is increased with a signifcant amount e.g. larger unmodelled uncertainty, the Lyaponov Redesign controller seems more robust, while the sliding mode controller target error is larger.